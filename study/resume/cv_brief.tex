\documentclass[titlepage]{article}
\usepackage{xeCJK}
\usepackage{fontspec}  
\usepackage{fancyhdr}  
\usepackage{indentfirst} 
%\usepackage[xetex]{hyperref}
%\usepackage{hyperref}
\usepackage{color}
\usepackage{listings}
\usepackage{xcolor}
\usepackage{graphicx}

\usepackage{paralist}
\usepackage[xetex,
            bookmarksnumbered,
            bookmarksopen,
            colorlinks,
%            urlbordercolor={111 111 111},
%            citecolor=blue,
            linkcolor=blue,
%            anchorcolor=blue,
            urlcolor=blue,
            ]{hyperref}
%\usepackage[CJKbookmarks]{hyperref}  

%\setlength{\parindent}{2em}
\addtolength{\hoffset}{-1cm}
\addtolength{\textwidth}{2cm}

\setCJKmainfont{STHeiti}
\setmainfont{Menlo} 
\begin{document}

%\lstset{numbers=left,
%numberstyle=\tiny,
%keywordstyle=\color{blue!70}, commentstyle=\color{red!50!green!50!blue!50},
%frame=shadowbox,
%rulesepcolor=\color{red!20!green!20!blue!20}
%}
\lstset{ 
frame=shadowbox, 
%numbers=left, 
%numberstyle=\tiny, 
escapeinside=``, 
xleftmargin=2em, 
xrightmargin=2em, 
aboveskip=1em,
rulesepcolor=\color{red!20!green!20!blue!20},
%rulesepcolor=\color{brown}
%keywordstyle=\color{blue!70}, 
%commentstyle=\color{red!50!green!50!blue!50}, 
keywordstyle=\color{blue}\bfseries,
commentstyle=\color{olive},
%directivestyle=\color{blue},
stringstyle=\color{purple},
%backgroundcolor=\color{gray},
%basicstyle=\tiny\ttfamily,
%basicstyle=\tiny,
%framerule=0pt,
breaklines=true,
}

\pagestyle{fancy}
\lhead{}
\lfoot{}
\cfoot{}
\rfoot{}
\renewcommand{\headrulewidth}{0.4pt}
\renewcommand{\footrulewidth}{0.4pt}

\title{个人简历} 
\author{金磊} 
\renewcommand{\today}{\number\year 年 \number\month 月 \number\day 日}
\date{\today} 
\maketitle

\setlength{\parindent}{2em} 
\addtolength{\parskip}{3pt}

%\pdfbookmark[2]{Bookmarktitle}{internallabel}
\renewcommand{\contentsname}{目录}
\setcounter{tocdepth}{2}
\tableofcontents 

\newpage
\section{基本情况}

\begin{tabular}{|l|l|}
\hline
姓名 & 金磊 \\
\hline
年龄 & 31 \\
\hline
毕业院校 & 天津大学 \\
\hline
学历 & 通信系统工程硕士 \\
\hline 
出生地 & 天津 \\
\hline 
手机 & 13820788636 \\
\hline 
住址 & 北京昌平 \\
\hline 
婚姻状况 & 已婚 \\
\hline 
邮箱 & 125981281@qq.com \\
\hline 
github & \href{http:/www.github.com/jinleileiking?tab=repositories}{github.com/jinleileiking} \\
\hline
\end{tabular}



\section{工作经历}

\begin{tabular}{|l|l|l|}
\hline
2006-2007 & 华为3com实习 & 软件工程师 \\
\hline
2007-2014 & 	中兴天津研究 & 	软件工程师,系统工程师 \\
\hline
2014.3-2014.6 & 	AmpedRf  & 	高级嵌入式软件工程师 \\
             &  http://www.ampedrftech.com &                         \\
\hline
2014.6-至今 & 百度 & 高级研发工程师 \\
\hline
\end{tabular}

\section{教育经历}

\begin{tabular}{|l|l|l|l|l|}
\hline
2004-2007 & 天津大学  & 研究生 &  主修神经网络 & 毕业设计 \\
          & 通信工程  &        & 现代编码与调制&“基于ARM S3C44B0的\\
          &           &        & 通信网理论    & 嵌入式uCLinux下\\
          &           &        & 数字信号处理  &的温控湿热室”\\
\hline
\end{tabular}


\section{工作项目经历}

\subsection{百度}

主要是从事新路由\url{http://newifi.baidu.com/}相关研发,基于openwrt,包括:

\begin{compactitem}
    \item 通过反编译,抓包等手段分析了小米路由器的预取原理
    \item 利用spdy虫洞技术,经过nginx实现,实现加速读取网页
    \item 利用预取技术,使用nginx+lua+memcache, nginx\_cache方案实现,最终用netfilter+netlink来实现。
    \item 利用netfilter+netlink分析包,结合lbs,使路由器可以由lbs获得位置信息。
    \item 使用netfilter+netlink, 结合百度云安全部门,完成云安全模块, 识别钓鱼,挂马网站
    \item 完成软件包实时升级模块
    \item 完成android的进程管理zygote的移植与相关功能开发
    \item 路由插件启停功能的开发。
    \item 完成了分支 merge
    \item 完成aria2移植
    \item 解决ubus超时问题
    \item 完成隐藏ssid功能
    \item 保存wan口状态功能
    \item dns server 相关: dnsmasq, unbound
    \item netpass,网络加速插件
    \item 完成qos调研
    \item 完成wifi sniffer模式的调研,wifi sniffer在混杂模式是否能给得到各种手机的mac的结论
    \item mqtt调研,选型
\end{compactitem}

参与智能家居项目开发

\subsection{AmpedRF}

将linux的wifi协议栈(内核+wpa\_supplicant)移植到freeRTOS+lwip小型嵌入式系统上,该设备实现airplay以便被苹果使用, 此项目中,我熟悉了linux网络协议栈相关的内容。以及wpa\_supplicant的代码

\subsection{中兴天津研究所}

研究生毕业,进入中兴天津研究所,从事RFID设备开发.

\subsubsection{中国移动移动支付项目}
在2009年左右,中国移动发起了移动支付RFSIM的项目, 我们作为参与方,我带队参与了整个项目的开发,测试. 在移动总部旁边的如家酒店进行了3个星期的奋战.整个项目包括PC 端 和设备端. 设备端包括 EC01 门禁控制器, 6602C读卡器, 6201 POS机. 我负责整体问题的解决. 和其他厂商的对接,参与了EC01设备 sqlite数据库问题的排除. 6201 RFSIM驱动的编写,调试. 等. 同时也熟悉了后台系统(cs+bs, 基于java)

\subsubsection{用ARM7TDMI CPU实现EPC C1G2协议}
制作一个EPC C1G2的阅读器 3600. 这个项目我负责完成EPC C1G2协议的实现.整个系统是搭建在中兴一个私有的操作系统上,设计的很简单,类似于ucos. 完成这个协议写了近10万行代码.协议比较复杂,有mac层,也有物理层. 这个项目设计的难点,就是物理层的实现. 用cpu实现物理层的高速传输,并进行物理层信号的过滤,比较困难,一般大都是采用fpga做物理层的一些工作.这次项目成本考虑,采用了arm. 在这方面下了很大的功夫,才最终搞定.也参与了一些驱动的实现.包括EEPROM, UART, ADC等.


\subsubsection{基于R2000的EPC阅读器}
此项目是基于TI AM1808+linux. 软件由我一人完成. 包括内核裁剪, 驱动(串口,i2c, spi, usb, eth, rs485, wlan, sd卡, usbhub),应用.由于对比evm板,更换了许多设备,所以改了很多地方. 应用层方面,实现了llrp协议.一个EPC方面的网口通信协议. 协议比较复杂. 用了很多线程. 完成了客户端和服务器端的设计,编码.


\subsubsection{完成其他rfid领域桌面式阅读器, 车载标签的开发}



\subsection{研究生华为3COM实习}

研究生阶段,在华三实习,实习的平台是私有的操作系统,这个操作系统实际上是封装了vxworks和psos两个操作系统的系统调用。实际工作是维护x.25协议栈。开发atm协议的新功能。完成lapb的新功能。给泰国项目添加了LAPB的T4功能.由于时间紧迫,当初是熬夜完成了这个工作.完成在MPLS L2VPN网络上对ATM信元进行透传,需要将ATM信元按照IETF的草案进行打包,透过MPLS网络进行透传。 完成在X2T报文中添加目的和源X.121地址的功能,用来代替博达公司的路由器。 

\section{个人性格}

\begin{compactitem}
    \item 勤奋,踏实
    \item {\color{red}stay hungry, stay foolish. Be a humble man}
    \item 希望有生之年能做出{\color{red}改变大众生活习惯的创造性产品}
    \item 爱学习,读书,喜欢编程
    \item 成为技术专家是我的梦想
    \item 爱音乐(古典,流行),足球
\end{compactitem}


\subsection{开源贡献}

\begin{compactitem}
    \item 为awesome写了几插件(Linux的wm, lua)
    \item 为subtle写了几个插件(Linux的wm, ruby)
    \item 用rails写过几个小网站
    \item 曾经的代码都放在我的github上
\end{compactitem}



\end{document}

